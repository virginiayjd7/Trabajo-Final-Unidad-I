
\documentclass[preprint,12pt]{elsarticle}

\usepackage[spanish]{babel}
\usepackage{amssymb}
\usepackage{graphicx}
\usepackage{lineno}
\usepackage[utf8]{inputenc}
\usepackage{url}
\usepackage{natbib}

\begin{document}
	
	\begin{frontmatter}

		\title{\huge  PROYECTO DE  ANALISIS Y  MEJORAMIENTO DE  SOFTWARE }
		\author{Yaneth Virginia Aquino Huallp                (2017059286)}
		\author{Jesus Alejandro Quenaya Buiza              (---)}
		\author{Daniel Angel Juarez Chumpe              (---)}
		\address{Tacna, Perú}
		


%%INICIO abstract
\begin{abstract}
Con el presente documento se pretende presentar de forma concisa y clara las necesidades del cliente en el área a de ventas en términos de software que se realizará. 
En esta documentación se plasmará los requerimientos que servirán de guía para desarrollar el software en sus distintas etapas, ayudándonos a validar e inspeccionar la construcción de este, aplicando la calidad de Software.
Por ello se trabajara con  Sonarqube para el análisis de código estático
para analizar el código y encontrar errores de código, y vulnerabilidades de seguridad. 
El análisis C # de SonarSource tiene una gran cobertura de estándares de calidad bien establecidos.  
\end{abstract}
%%FIN abstract


\end{frontmatter}
%%INICIO Introducción
\section{Antecedentes oIntroducción}

El sistema de "SysFerreteria" esta desarrollado para poder automatizar los procesos que
normalmente realiza de forma manual, esto para llevar un correcto control del
inventario, compras a proveedores y ventas que se realizan a diario a los clientes;Además, implementar la emisión de Comprobantes de Pago para  los clientes.
%%FIN Introducción

%%INICIO Titulo
\section{Sistema de Ferreteria}
Propuesta de un sistema de facturación de  compra y venta
de artículos de ferretería.
%%FIN Titulo
%%INICIO Autores
\section{.................}
%%FIN Autores
%%INICIO Planteamiento del problema
\section{Planteamiento del problema}
En la actualidad, para las empresas resulta una obligación llevar un mejor control y resguardo
de la información ya que de ésta depende, en su mayoría, el aumento o disminución de sus
ganancias o, a su vez, permite que las gestiones realizadas por sus trabajadores sean lo más
eficientes y óptimas para de ésta manera lograr los resultados esperados.

%%----------------------------------------------------------------------------------------------------------------------------------------------------------
	\subsection{\textbf{Problema}}
 En la microempresa “fffffff” ningún proceso se encontraba automatizado, todas
las actividades se realizaban manualmente y no se lograban concretar los procesos debido a la
magnitud de los mismos; por tal razón se procederá a desarrollar el sistema que contará con
módulos de usuarios, módulo de facturación, módulo de compras, gestión de inventario, módulo
de consultas y la generación de reportes que cumplan con los requerimientos de los usuarios
 y términos.
%%-----------------------------------------------------------------------------
	\subsection{\textbf{Justificación }}
Con el  presente proyecto  se pretende desarrollar el sistema de facturación para que de ésta forma el propietario conozca exactamente las ganancias netas diarias o mensuales
dependiendo de la necesidad que se tenga; además la falta de gestión del inventario impide al
usuario conocer las cantidades exactas de los productos que mantiene, su costo real y si es
necesario o no realizar pedidos en los tiempos determinados.

%%-----------------------------------------------------------------------------
	\subsection{\textbf{ Alcance }}
El software a desarrollar se basará en el área de ventas de la empresa, usando también el stock de los productos por lo cual:
El sistema permitirá registrar, actualizar, buscar y dar por “deteriorado” la información de los productos en stock. 
El sistema realizará ventas la cual nos permitirá agregar productos a la venta, restar con el stock, registrar o anular venta. 
El sistema permitirá añadir pedido (datos del producto y del cliente), registrar o anular pedido y listar pedidos.
El sistema permitirá generar reportes sobre: Ventas del día con el total ganado y Lista de Pedidos.
El sistema permitirá realizar la autentificación de los usuarios


\section{Objetivos}
		\subsection{\textbf{ General }}
	 \begin{itemize}
		\item Desarrollar un software enfocado en el área de compra y ventas de la microempresa.
	 \end{itemize}
		\subsection{\textbf{Específicos }}
\begin{itemize}
	\item El sistema permitirá registrar, actualizar, buscar y dar por “deteriorado” la información de los productos en stock.
	\item El sistema realizará ventas la cual nos permitirá agregar productos a la venta, restar con el stock, registrar o anular venta. 
	\item El sistema permitirá añadir pedido (datos del producto y del cliente), registrar o anular pedido y listar pedidos. 
	\end{itemize}

	\section{Referentes teóricos}

	\section{Desarrollo de la propuesta}

		\subsection{\textbf{Tecnología de información  }}
		
		\subsection{\textbf{ Metodología, técnicas usadas  }}
\section{Cronograma }
	
	\newpage
	\bibliographystyle{apalike}
	\bibliography{BIBLIOGRAFIA}	
%\citep{referenciarobles2}  


% https://revistas.udistrital.edu.co/index.php/tia/article/view/4639/7094
% http://revistas.unitru.edu.pe/index.php/PGM/article/view/193/199
% http://www.spentamexico.org/v4-n2/4(2)%2016-52.pdf          ---> PAGINA 18


\end{document}

